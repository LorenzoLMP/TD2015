%\documentclass[a4paper]{article}
%\usepackage[T1]{fontenc}
%\usepackage[utf8]{inputenc}
%\usepackage[italian]{babel}
%\usepackage{amssymb}
%\usepackage{amsmath}
%\usepackage{hyperref}
%\usepackage{amsthm}
%\usepackage{graphicx}
\documentclass[journal, a4paper]{IEEEtran}
\usepackage[italian]{babel}
\usepackage{booktabs}
\usepackage{siunitx}%Questo serve a caricare il pacchetto delle unità di misura del sistema internazionale%
\usepackage[utf8]{inputenc}
\usepackage{graphicx} 
\usepackage{url}
\usepackage{amsmath}


\usepackage{keyval}
\usepackage{xcolor}
\usepackage{caption}
\usepackage{tikz}
\usepackage{circuitikz}
\usepackage{authblk}
%\usepackage{hyperref}

\begin{document}


% Define document title and author
	\title{Tecnologie Digitali - Logbook Week 2}
	\author[1]{Salvatore Bottaro}
		\author[2]{Lorenzo M. Perrone}
		\affil[1]{\texttt{email@sa.com}}
		\affil[2]{\texttt{lorenzo.perrone.lmp@gmail.com}}
	\markboth{Tecnologie Digitali - Di Lieto}{}
	\maketitle
	
\begin{abstract}
	Logbook di laboratorio di Tecnologie Digitali, a.a. 2015/2016. Week 2.
\end{abstract}

\section{Lezione 06/10/2015}
Durante la lezione di oggi, è stato introdotto il funzionamento dell' \textit{amplificatore operazionale} (detto \textsc{op-amp}). Tale componente fu ideato nel 1934 dall'ignegnere della Bell, Black H., che stava cercando un modo per amplificare i segnali telefonici (mantenendo un guadagno il più possibile uniforme fra le frequenze tipiche dello spettro uditivo $10\si{Hz}\rightarrow 10 \si{kHz} $),  e soprattutto modulare questa amplificazione in base ai fattori esterni, come condizioni metereologiche o strumentali. La soluzione fu quella di introdurre un amplificatore sovrapotenziato da regolare tramite un circuito di reazione (\textit{feedback}) e dei componenti passivi.

\begin{circuitikz}
\centering

\end{circuitikz}



\begin{figure}[htp]

\end{figure}




%\begin{equation}

%\end{equation}


\subsection{}


\subsection{}



\section{Conclusion}
	This section summarizes the paper.

% Now we need a bibliography:
\begin{thebibliography}{5}

	%Each item starts with a \bibitem{reference} command and the details thereafter.
	\bibitem{HOP96} % Transaction paper
	J.~Hagenauer, E.~Offer, and L.~Papke. Iterative decoding of binary block
	and convolutional codes. {\em IEEE Trans. Inform. Theory},
	vol.~42, no.~2, pp.~429–-445, Mar. 1996.

	\bibitem{MJH06} % Conference paper
	T.~Mayer, H.~Jenkac, and J.~Hagenauer. Turbo base-station cooperation for intercell interference cancellation. {\em IEEE Int. Conf. Commun. (ICC)}, Istanbul, Turkey, pp.~356--361, June 2006.

	\bibitem{Proakis} % Book
	J.~G.~Proakis. {\em Digital Communications}. McGraw-Hill Book Co.,
	New York, USA, 3rd edition, 1995.

	\bibitem{talk} % Web document
	F.~R.~Kschischang. Giving a talk: Guidelines for the Preparation and Presentation of Technical Seminars.
	\url{http://www.comm.toronto.edu/frank/guide/guide.pdf}.

	\bibitem{5}
	IEEE Transactions \LaTeX and Microsoft Word Style Files.
	\url{http://www.ieee.org/web/publications/authors/transjnl/index.html}

\end{thebibliography}

% Your document ends here!
\end{document}

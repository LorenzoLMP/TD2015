%\documentclass[a4paper]{article}
%\usepackage[T1]{fontenc}
%\usepackage[utf8]{inputenc}
%\usepackage[italian]{babel}
%\usepackage{amssymb}
%\usepackage{amsmath}
%\usepackage{hyperref}
%\usepackage{amsthm}
%\usepackage{graphicx}
\documentclass[journal, a4paper]{IEEEtran}
\usepackage[italian]{babel}
\usepackage{booktabs}
\usepackage{siunitx}%Questo serve a caricare il pacchetto delle unità di misura del sistema internazionale%
\usepackage[utf8]{inputenc}
\usepackage{graphicx} 
\usepackage{url}
\usepackage{amsmath}


\usepackage{keyval}
\usepackage{xcolor}
\usepackage{caption}
\usepackage{tikz}
\usepackage{circuitikz}
\usepackage{authblk}
%\usepackage{hyperref}

\begin{document}


% Define document title and author
	\title{Tecnologie Digitali - Logbook Week 1}
	\author[1]{Salvatore Bottaro}
		\author[2]{Lorenzo M. Perrone}
		\affil[1]{\texttt{salvo.bottaro@hotmail.it}}
		\affil[2]{\texttt{lorenzo.perrone.lmp@gmail.com}}
	\markboth{Tecnologie Digitali - Di Lieto}{}
	\maketitle
	
\begin{abstract}
	Logbook di laboratorio di Tecnologie Digitali, a.a. 2015/2016. Week 1.
\end{abstract}

\section{Lezione 28/09/2015}
Abbiamo misurato la d.d.p. $V_{out}$ ai capi della resistenza $R_2$ di un partitore di tensione secondo il seguente schema circuitale:\\

\begin{circuitikz}
\centering
\draw (0 ,0) node[anchor=east] {$V_{in}$};
\draw (0 ,0) to[short](1.5,0);
\draw (1.5,0) to[R,l^=$R_1$](1.5, -1.5);
\draw (1.5, -1.5) to[short, o-](3, -1.5);
\draw (3, -1.5) node[anchor=west] {$V_{out}$};
\draw (1.5, -1.5) to[R,l^=$R_2$](1.5, -3);
\node[ground]at (1.5 , -3){};
\end{circuitikz}


dove $V_{in}$ è la tensione in ingresso.\\
Sia il segnale in ingresso che l'analisi del segnale in uscita sono stati ottenuti per mezzo del VI \textit{v\_in\_v\_out.vi} composto da tre pannels. Il primo pannel contiene una copia del VI \textit{v\_in\_v\_out.vi} che genera segnali su un fondoscala di 10 V con profondità digitale di 12 bit (dunque con una risoluzione di 5 mV), il secondo pannel contiene un VI per ritardare di qualche ms l'acquisizione del segnale in uscita rispetto all'istante in cui viene generato il segnale in ingresso, il terzo pannel contiene il VI per l'analisi del segnale in uscita e che restituisce sul front pannel il valor medio sui campionamenti effettuati e relativa deviazione standard.\\

Il circuito è stato realizzato sulla breadboard prendendo come resistenze $R_1 = 22 ~ k\Omega \pm 10 \%$ e $R_2 = 220~ k\Omega \pm 10 \%$, scelte in modo da garantire che la corrente nel circuito fosse dell'ordine del $\mu$A. Il cavo per la terra è stato collegato alla CB29 della scheda di acquisizione, il cavo in $V_{in}$ alla CB22, analog output 0 della scheda, mentre il cavo in $V_{out}$ alla CB68, analog input 0 della scheda. \\
Abbiamo scelto $V_{in} = 2.75 V$, dopodiché prima di collegare la breadboard alla scheda abbiamo verificato il corretto funzionamento della scheda collegando il CB22 al CB68 e avviando il VI. Il valore restituito è stato $V_{out} = 2.749 V \pm 0.001 V$ che garantisce il corretto funzionamento della scheda. Abbiamo infine collegato la scheda al circuito e avviato il VI. Il valore atteso si ottiene da:\\

\begin{equation}
V_{out}^{att}= \frac{V_{in}}{1+\frac{R_1}{R_2}}
\end{equation}\\

da cui, per la scelta delle resistenze, $V_{out}^{att} = 2.5 V \pm 20 \%$. Il valore registrato è stato $V_{out} = 2.502 V \pm 0.002 V$. Abbiamo scambiato il cavo per la CB29 con quello della CB22, scambiando così i ruoli delle resistenze. In tal caso si ha $V_{out}^{att} = 0.25~ V \pm 20 \%$ mentre quello registrato $V_{out} = 0.244 ~V \pm 0.001 ~V$.

\section{Lezione 29/09/2015}
Durante la lezione odierna, abbiamo iniziato a lavorare con il software TINA, un simulatore analogico SPICE-Based prodotto dalla \textit{Texas Instruments}. Tramite TINA è possibile analizzare il comportamento di circuiti più o meno complessi, potendo inserire numerosi componenti circuitali di cui settare i valori. \\
In primo luogo è stato riprodotto il circuito impiegato nella lezione precedente e analizzato in continua per verificare che i valori da noi trovati fossero compatibili con quelli teorici previsti da TINA, usando la funzione \textsc{dc transfer characteristic}.
Quindi abbiamo spostato la nostra attenzione sui circuiti in alternata, per i quali è possibile determinare una \textit{funzione di trasferimento} $V_{out} = f(V_{in})$, dove $V_{in}$ è la tensione (alternata) fornita in ingresso, e $V_{out}$ quella in uscita.
I circuiti che esamineremo prevalentemente saranno di tipo \textit{lineare}, dove è possibile stabilire una relazione fra $V_{in}$ e $V_{out}$ come segue:

\begin{equation}
V_{out}(\omega)= H(\omega)V_{in}(\omega)
\end{equation}


\section{Lezione 30/09/2015}
Proseguendo il discorso avviato durante la lezione precedente, tramite un VI apposito (bode\_ph.vi), si è analizzato l'andamento della funzione di trasferimento attraverso un diagramma di Bode per un circuito strutturato come segue (e realizzato sulla \textit{Breadboard}):\\


\begin{figure}
\centering
\includegraphics[width=0.9\linewidth]{./breadboard1_passabasso}
\caption{Circuito passa basso realizzato sulla breadboard.}
\label{fig:breadboard_1}
\end{figure}


\begin{itemize}
\item Resistenza: $22 k\Omega \pm 5 $ \% tolleranza.
\item Condensatore: $68\si{nF}$, con permittività 10, di polipropilene.\\
\end{itemize}


Per questo circuito è prevista una frequenza di taglio di $f_T =\frac{1}{2\pi RC} = 106.4 \si{Hz}$.

All'interno del \textit{VI} è possibile impostare i valori iniziali per il range di acquisizione delle frequenze, il numero di misure (campionamento), il numero di periodi per ogni sinusoide e infine il fondoscala. I valori da noi usati per questa prima acquisizione sono:
\\

\begin{tabular}{|c|c|}
\hline range frequenze & 5-2000 \si{Hz} \\ 
\hline num. campionamenti & 30 \\ 
\hline num. periodi & 5 \\ 
\hline num. camp/periodo & 25 \\ 
\hline fondoscala & 5 \si{V} \\ 
\hline 
\end{tabular} \\


Ci aspettiamo che il comportamento del circuito sia quello di un filtro passa-basso (passivo), che tagli, cioè, le frequenze maggiori della frequenza di taglio, e lasci passare (attenuando poco) quelle minori di tale soglia. \\
I grafici ricostruiti da \textit{LabView} mostrano un andamento simile a quanto aspettato. In particolare viene chiesto quale potrebbe essere l'attenuazione (con tensione in ingresso normalizzata ad 1) per una frequenza di $10 \si{kH}$. Si può rispondere a questa domanda in più modi: partendo da un esame prettamente grafico notiamo che alla frequenza di circa $2000\si{kHz}$, l'attenuazione è circa $0.05$, in diminuzione per questo range di frequenze. Interpolando i punti sperimentali e prolungando la curva tracciata, risulta plausibile un valore dell'attenuazione dell'ordine di $10^{-2}$. \\
Alternativamente, possiamo analizzare il circuito scrivendo il modulo della funzione di trasferimento del circuito e la formula per lo sfasamento introdotto fra $V_{in}$ e $V_{out}$, che sono come segue:

\begin{equation}\label{F_trasf}
\rvert H(f) \lvert = \frac{1}{\sqrt{1+(\frac{f}{f_T})^2}}
\end{equation}

\begin{equation}
\Delta \phi = (-)\arctan(\frac{f}{f_T})
\end{equation}


E' interessante studiare il limite asintotico per alte frequenze (alte rispetto alla frequenza di taglio), che ci fornisce un'ottima approssimazione del valore richiesto.

\begin{equation}
\rvert H(f) \lvert = \frac{1}{\sqrt{1+(\frac{f}{f_T})^2}}
\sim \frac{f_T}{f} 
= \frac{106.4 \si{Hz}}{10\si{kHz}} 
\approx 10 mV 
%(\mbox{per f \gg f_T})
\end{equation}


assolutamente compatibile con l'esame qualitativo precedente. Che il modello teorico risulti in accordo con i dati sperimentali (per $f \gg f_T$), lo si può vedere esaminando il grafico delle attenuazioni in scala bilog: se effettivamente l'andamento funzionale deve essere come $1/f$, ci aspettiamo che in scala bilog risulti una retta con pendenza circa $-1$, che è quanto risulta dal fit dei dati.

\begin{figure}
\centering
\includegraphics[width=1.1\linewidth]{./attenuaz_passa_basso_loglog}
\caption{Attenuazione in scala bilog del filtro passa-basso}
\label{fig:attenuaz_passa_basso_loglog}
\end{figure}



Cambiando i valori dei parametri di input nel \textit{VI} è stato osservato che la misura dello sfasamento diventa sempre meno accurata man mano che viene ridotto il numero di campionamenti per periodo. Una possibile spiegazione di questo fenomeno è che se i punti sperimentali sono pochi, diventa sempre più difficile per il calcolatore tracciare una forma d'onda precisa, di cui si possa calcolare lo sfasamento rispetto al segnale in ingresso.

Per le frequenze $f_{50} = 50 Hz$ e $f_{500} = 500 Hz$ i valori dell'attenuazione e dello sfasamento previsti dal modello sono riportati nella tabella seguente:\\

\begin{tabular}{|c|c|c|}
\hline  & $f_{50}$ & $f_{500}$ \\ 
\hline Attenuazione & $0.89$ & $0.19$ \\ 
\hline $\Delta \phi $& $(-)26$  & $(-)78$ \\ 
\hline 
\end{tabular} \\


\section{Analisi AC con TINA}
A questo punto, tramite il software \textsc{TINA}, è stato riprodotto il circuito da noi costruito sulla breadboard e sono state settate come frequenze per il generatore AC proprio  $f_{50}$ e $f_{500}$, per cui \textsc{TINA} ha gentilmente calcolato i valori dell'attenuazione e dello sfasamento per via simbolica. \\

\begin{tabular}{|c|c|c|}
\hline  & $f_{50}$ & $f_{500}$ \\ 
\hline $V_{in}$ & $1 \si{V}$ & $1 \si{V}$ \\ 
\hline $\Delta \phi $& $(-)25.17$  & $(-)77.99$ \\ 
\hline Tensione su C & $905 \si{mV}$ & $208.12 \si{mV}$ \\ 
\hline 
\end{tabular} \\

Cambiando la scala da logaritmica in lineare, il grafico restituito da \textsc{TINA} assume una forma a campana centrata sul primo valore della frequenza. In questa scala è possibile studiare meglio gli andamenti per basse frequenze che ci aspettiamo quadratico, e infatti sviluppando al primo termine non nullo in $x= f/f_T$, risulta in una parabola con la concavità rivolta verso il basso. \\
Gli andamenti per alte frequenze sono più chiaramente leggibili in scala logaritmica, in cui l'andamento lineare è palese.

Per la regione del grafico compresa fra $500-1000 \si{Hz}$ il guadagno di $-6db$ dimezzando la frequenza è perfettamente verificato. Altresì, tale guadagno viene mostrato alla frequenza di $180\si{Hz}$ e l'amplificazione del segnale è di $0.5$. Questa frequenza assume un'importanza particolare poichè, a partire dalla (\ref{F_trasf}), ponendo l'amplificazione pari a $1/2$ la $f_{1/2}$ alla quale questa si verifica è $f_{1/2}^{exp} = \sqrt{3}f_T = 184 \si{Hz} \approx f_{1/2} = 180 \si{Hz} $.\\

Come controprova sul circuito realizzato sperimentalmente, impostando una frequenza di start dello sweep $f_{start} = 160 \si{Hz}$ e una frequenza di end-sweep $f_{end} = 200 \si{Hz}$, otteniamo che in corrispondenza di un'amplificazione di $0.5$, si ha una frequenza pari circa a $f = 182.36 \si{Hz}$, in linea con quanto visto.

\subsection{Approfondimento: Voltage generator di TINA - Homework 1}
Nella finestra del \textit{Voltage Generator} di TINA si può scegliere il tipo di forma d'onda del segnale generato, e ovviamente i parametri necessari affinchè ciascuna forma sia ben definita, come frequenza, fase e ampiezza per quelle sinusoidali, il \textit{rise time} per quelle quadrate, l'ampiezza dell'impulso per i treni di impulsi ecc.\\
E' possibile, inoltre, scegliere un offset DC aggiunto, quindi, al segnale AC, così come impostare una resistenza interna al generatore, facendolo diventare "reale" da "ideale".


\subsection{Filtro passa-alto con TINA - Homework 2}
L'altra tipologia di filtro che si può realizzare impiegando componenti capacitivi e resistivi è il filtro passa-alto, che si ottiene a partire dal filtro passa-basso scambiando nel circuito condensatore e resistenza. Come suggerisce il nome, un circuito di questo tipo taglia le frequenze inferiori alla frequenza di taglio (definita allo stesso modo che per il passa-alto), e lascia passare con attenuazione 1 (cioè non attenua) quelle molto maggiori della $f_T$.\\
Valgono delle considerazioni analoghe a quanto fatto per il filtro passa-basso, con l'opportuna differenza che lo sfasamento del segnale è questa volta positivo. Nella (\ref{F_trasf_alto}) è riportato il modulo della fuunzione di trasferimento.\\

\begin{equation}\label{F_trasf_alto}
\rvert H(f) \lvert = \frac{1}{\sqrt{1+(\frac{f_T}{f})^2}}
\end{equation}

Si riporta in Figura(\ref{fig:passa_alto}) un grafico \textit{bode} realizzato con TINA, in cui i valori del condensatore e della resistenza scelti sono quelli che che verranno utilizzati nella sezione successiva, e quindi: $R = 100 \si{kOhm}$, $C = 47 \si{nF}$, e $f_T = 33 \si{Hz}$.\\

\begin{figure}
\centering
\includegraphics[width=0.9\linewidth]{./passa_alto}
\caption{Simulazione con Tina del funzionamento di un filtro passa-alto. Diagramma Bode.}
\label{fig:passa_alto}
\end{figure}




\section{Filtri passa-banda capacitivi}
Passiamo ora ad analizzare un filtro passa-banda capacitivo, composto da un filtro passa-basso collegato ad un filtro passa-alto. Riportiamo in Figura (\ref{fig:breadboard2_passabanda}) lo schema del circuito. \\

Sappiamo già che per avere un buon funzionamento della serie dei filtri è necessario che l'impedenza (vista dalla resistenza da $ 1 k\Omega $ ) del ramo contenente il condensatore da $ 100 nF $ deve essere molto minore di quella del ramo contenente il filto passa alto. Simbolicamente questo corrisponde a scrivere la condizione:

\begin{equation}
\frac{1}{j\omega C_2} + R_2 \gg \frac{1}{j\omega C_1}
\end{equation}

che nel nostro caso è ben verificata. Inoltre è necessario che le frequenze di lavoro siano molto minori di quella di taglio del passa basso e maggiori di quella del passa alto.

\begin{figure}
\centering
\includegraphics[width=0.7\linewidth]{./breadboard2_passabanda}
\caption{Schema del circuito passa-banda: date le capacità disponibili, C1 (inizialmente prevista di 100nF) è stata resa con un parallelo fra due capacità di 56nF ciascuna.}
\label{fig:breadboard2_passabanda}
\end{figure}


Le frequenze di taglio dei filtri sono:\\

\begin{tabular}{|c|c|}
\hline $f_{TA}$ & $f_{TB} $\\ 
\hline $1591 \si{Hz}$ & $33 \si{Hz}$ \\ 
\hline 
\end{tabular} \\

Poichè si tratta di una composizione di due filtri, lo sfasamento risulta essere in prima battuta la somma fra gli sfasamenti (uno è negativo e uno positivo), che in particolare risulta essere pari a zero per $f_0 = \sqrt{f_{T,A}f_{T,B}} $. Quindi, $f_0^{exp} = 232.11 \si{Hz}$, mentre quella interpolata graficamente dal diagramma \textsc{BODE} realizzato da \textsc{TINA}, è circa $230 \si{Hz}$. 

\begin{figure}
\centering
\includegraphics[width=1.1\linewidth]{./passabanda_sfasa_octave}
\caption{Grafico dell'attenuazione e dello sfasamento prodotto dal VI, acquisendo i dati dalla breadboard}
\label{fig:passabanda_sfasa_octave}
\end{figure}


Si può notare dalla figura (\ref{fig:passabanda_sfasa_octave}) acquisita con il \textit{VI}, che la regione in cui il guadagno è alto risulta piuttosto ampia, con frequenze appartenenti all'intervallo $80-800 \si{Hz}$. In altre parole, la larghezza del picco è grande. \\
Inoltre, per basse frequenze notiamo dei punti sperimentali sensibilmente scostati dalla spezzata prodotta dai restanti campionamenti. Per altre frequenze,invece, si evidenzia un buon accordo con l'andamento simulato con \textsc{Tina} riportato in Figura (\ref{fig:passabanda_sfasa_tina}). 

\begin{figure}
\centering
\includegraphics[width=1.1\linewidth]{./passabanda_sfasa_tina}
\caption{Grafico dell'attenuazione e dello sfasamento simulato da TINA per il circuito realizzato sulla breadboard}
\label{fig:passabanda_sfasa_tina}
\end{figure}

\subsection{Homework 3}

Per il filtro passa banda è possibile calcolare analiticamente la funzione di trasferimento. Facendo riferimento alla Figura \ref{fig:breadboard2_passabanda}, si ha per la legge dei nodi:
\begin{equation}
I_{in} = I_C + I_{CR}
\end{equation}
dove $I_{in}$ è la corrente che scorre lungo $R_1$, $I_C$ lungo $C_1$, $I_{CR}$ quella lungo l'altro ramo del parallelo. Se si indica con V la tensione al nodo fra $R_1$ e i due condensatori, si ha:
\begin{equation}
\frac{V_{in}-V}{Z_{R1}} = \frac{V}{Z_{C2}} + \frac{V}{Z_{C2}+Z_{R2}}
\end{equation}
mentre:
\begin{equation}
V_{out} = \frac{VZ_{R2}}{Z_{C2}Z_{R2}}
\end{equation}
da cui esplicitando i valori delle impedenze $Z_{Ri} = R_i$ e $Z_{Ci}=\frac{1}{j\omega C_i}$ si ottiene:
\begin{equation}
T(\omega) = \frac{j\omega C_1R_2}{(1+j\omega C_1R_2)(1+j\omega R_1C_1+\frac{j\omega C_2R_2}{1+j\omega C_2R_2})}.
\end{equation}
Per trovare la frequenza di massimo trasferimento, nel nostro caso si può sfruttare il fatto che $R_2 \gg R_1$, ottenendo:
\begin{equation}
T(\omega) = \frac{j\omega C_1R_2}{(1+j\omega C_1R_2)(1+j\omega R_1C_1)}.
\end{equation}
Massimizzando il logaritmo del modulo si ottiene $\omega _{max} = \frac{1}{\sqrt{R_1R_2C_1C_2}} \propto \frac{1}{\sqrt{R_2}}$.
Per i componenti impiegati si ha $\omega _{max} = 218 Hz$. Dalla simulazione con TINA si ottiene 211 Hz, circa il 3 $\%$ più piccolo del valore calcolato, per cui l'approssimazione è abbastanza buona. \\
Un dispositivo con bassa impedenza in ingresso potrebbe creare dei problemi poiché tale resistenza andando in parallelo con $R_2$ ed essendo questa molto maggiore, molta della corrente scorrerebbe nel dispositivo anziché in $R_2$ che pertanto si può trascurare, cambiando completamente la risposta del filtro. Se invece l'impedenza in ingresso del dispositivo è molto alta, esso non influenzerebbe molto il funzionamento del filtro, per cui la sua risposta rimarrebbe inalterata.

\section{Circuito filtro passa-banda RLC serie}

\paragraph{Es. 11}
Abbiamo realizzato con TINA il seguente circuito:

\begin{circuitikz}
\centering
\draw (0.5 ,0) node[anchor=east] {$V_{in}$};
\draw (0.5 ,0) to[short, o-](1.5,0);
\draw (1.5,0) to[R,l^=$R$](1.5, -1.5);
\draw (1.5, -1.5) to[L,l^=$L$](1.5, -3);
\draw (2.5 ,-3) to[short, o-](1.5,-3);
\draw (2.5, -3) node[anchor=west] {$V_{out}$};
\draw (1.5, -3) to[C,l^=$C$](1.5, -4.5);
\node[ground]at (1.5 , -4.5){};
\end{circuitikz}

Sono stati assegnati ai vari componenti i seguenti valori: \textit{L=$4.7$ mH$,$~C=$100$ nF$,$~R=$10~\Omega$}, da cui lo schema in figura:

\begin{figure}[htp]
\centering

\includegraphics[scale=.4]{RLC}

\caption{Circuito realizzato con TINA}
\end{figure}

Abbiamo avviato la simulazione mandando in ingresso un segnale sinusoidale di ampiezza pp di 2 V, spazzando un intervallo di frequenze (1 $Hz$, 20 $kHz$). I grafici per l'ampiezza e per lo sfasamento restituiti sono:

\begin{figure}[htp]
\caption{Grafici della simulazione del circuito RLC secondo i valori fissati realizzato con TINA. La linea verde rappresenta i riferimento del segnale in ingresso. Il grafico superiore rappresenta il guadagno, quello inferiore lo sfasamento, entrambi in scala lineare.}
\label{fig:fig1}
\centering
\includegraphics[scale=.4]{tinadiag}
\end{figure}

La curva relativa al guadagno risulta piccata intorno al valore $f=7.35 kHz$, per cui l'amplificazione è circa 25 dB, con una FWHM di circa 1.5 $kHz$ e si mantiene al di sopra della linea verde, ovvero il circuito amplifica il segnale in ingresso, fino alla frequenza di 10 kHz, dopodiché il circuito comincia ad attenuare il segnale in ingresso.

\paragraph{Es. 12}
Dai valori della simulazione si ha che $f_0 = 7.34 kHz$, che nel grafico corrisponde al punto in cui si ha l'amplificazione massima, dunque $f_0$ rappresenta la frequenza di risonanza de circuito.\\
Per capire come variasse l'ampiezza della curva in funzione dei parametri del circuito, abbiamo effettuato diverse simulazioni variando i valori dei componenti. I risultati sono raccolti in
\ref{tab:tab1}.

\begin{table}[htp]
\caption{}
\label{tab:tab1}
\begin{tabular}{c|c|c|c}
\hline 
R ($\Omega$) & L ($mH$) & C ($nF$) & FWHM ($kHz$) \\ 
\hline 
10 & 4.7 & 100 & 1.5 \\ 
\hline 
20 & 4.7 & 100 & 2 \\ 
\hline 
30 & 4.7 & 100 & 2.6 \\ 
\hline 
40 & 4.7 & 100 & 3 \\ 
\hline 
10 & 9.4 & 100 & 0.9 \\ 
\hline 
10 & 18.8 & 100 & 0.6 \\ 
\hline 
10 & 37.6 & 100 & 0.45 \\ 
\hline 
10 & 4.7 & 25 & 2 \\ 
\hline 
10 & 4.7 & 50 & 1.8 \\ 
\hline 
10 & 4.7 & 200 & 1.2 \\
\hline
\end{tabular} 
\centering
\end{table}

Dai dati si nota che l'ampiezza del grafico e dunque la selettività in frequenza aumenta all'aumentare della resistenza, mentre diminuisce all'aumentare dell'induttanza e della capacità.

\paragraph{Es. 13}
In tabella ~\ref{tab:tab2} i dati relativi all'induttore impiegato per la realizzazione del circuito trovati sul sito della \texttt{Epcos}:

\begin{table}[htp]
\caption{}
\label{tab:tab2}
\centering
\begin{tabular}{c|c}
\hline
\textbf{Design} & "radial"\\
\hline
\textbf{Inductance} & 4.7 mH\\
\hline
\textbf{Tolerance} & 5 \% \\
\hline
\textbf{Rated Current} & 0.055 A\\
\hline
\textbf{Resistence} & 78 $\Omega$\\
\hline
\textbf{Quality factor} & 35\\
\hline
\textbf{Resonance frequency} & 0.7 MHz\\
\hline
\end{tabular}
\end{table}

\paragraph{Es. 14}

Abbiamo aggiunto una resistenza di $78 \Omega$ all'induttore, dopodiché abbiamo avviato un'altra simulazione con \textsc{TINA}. Il grafico restituito è il seguente:
\begin{figure}[htbp]
\centering
\includegraphics[scale=.4]{tinadiagresint}
\caption{Simulazione con resistenza interna nell'induttore.}
\end{figure}

Dal confronto con \ref{fig:fig1} appaiono evidenti delle differenze. Anzitutto la frequenza di risonanza si è abbassata a circa $7 Hz$, l'ampiezza è aumentata a $4 \si{kHz}$ (come era prevedibile per le prove dell'esercizio 12), ma soprattutto il massimo di amplificazione si è abbassato a $7.9 \si{dB}$, dovuto ovviamente alla maggiore dissipazione nel circuito.

\paragraph{Es. 15}
Abbiamo realizzato praticamente il circuito sulla breadboard, impiegando un resistore da 10 $\Omega$, un induttore da 4.7 mH e un parallelo di due condensatori da 56 nF, dunque con capacità equivalente di 112 nF. Abbiamo adattato le simulazioni di TINA a quest'ultimo valore della capacità. Abbiamo collegato la breadboard alla scheda di acquisizione e avviato la presa dati tramite il VI 'bode.ph'. L'andamento dei dati sperimentali si discosta sensibilmente da quello previsto dalle simulazioni con TINA, come si evince dal grafico seguente:

\begin{figure}[htp]
\centering
\includegraphics[scale=.34]{graph1}
\caption{Confronto dei dati sperimentali con l'andamento simulato. Gain in scala lineare}
\end{figure}

\begin{figure}[htp]
\centering
\includegraphics[scale=.34]{graph4}
\caption{Confronto dati sperimentali con andamento simulato con L = $4.8~mH$, C = $117 ~nF$ e resistenza interna dell'induttore a 74 $\Omega$. Gain in scala lineare}
\label{last}
\end{figure}

Come si vede, sebbene l'andamento del grafico sia simile a quello dei dati sperimentali, la frequenza di risonanza non corrisponde a quella osservata. Abbiamo imputato tale discrepanza ad un'eventuale imprecisione dei valori nominali dei componenti a disposizione, per cui abbiamo fatto varie simulazioni con TINA cambiando i valori di induttanza e capacità. Si riporta il confronto dei dati sperimentali con la simulazione in cui si è posto L = $4.8~mH$, C = $117 ~nF$ e la resistenza interna dell'induttore a 74 $\Omega$.
, in Figura (\ref{last}).



Si vede come il grafico riproduce molto più fedelmente i dati sperimentali. I valori impiegati nelle simulazioni rientrano nelle incertezze nominali riportati nei data sheet dei vari componenti, per cui è ragionevole aspettarsi che il motivo per cui i dati sperimentali si discostano dal grafico simulato sia da ricercarsi in valori effettivi dei parametri dei componenti che seppur non coincidenti con i rispettivi valori nominali rientrino nelle incertezze indicate.


% Now we need a bibliography:


% Your document ends here!
\end{document}